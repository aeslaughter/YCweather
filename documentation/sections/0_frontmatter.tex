\msu{\msusection{Preface}}{\section*{Preface}\markboth{Preface}{}\addcontentsline{toc}{section}{Preface}}
The program discussed in this user manual, YCweather, was produced by Andrew E. Slaughter at Montana State University (MSU) while completing a PhD in Applied Mechanics and researching at the Sub-zero Science and Engineering Research Facility (SSERF). YCweather was written as a requirement for completion of the degree program for Andrew Slaughter and is intended for use by researchers at SSERF or affiliated organizations.

\msusection{Overview}
The software discussed in this document, named YCweather, was created to allow for simple, fast access to weather station data, snow crystal images, and snow morphology information that was collected on a daily basis at the Yellowstone Club (YC) by the dedicated ski patrol.  YCweather also provides tools for examining and disseminating data.  It was designed to be generic in nature, such that it may be implemented by avalanche researchers and practitioners in Southwest Montana interested in accessing graphically local weather station data.  

The information contained in this document is meant to serve two purposes:
\begin{enumerate}
\item to provide a user manual for operating the software package as a standalone windows application, and
\item to provide sufficient details for future researchers at SSERF for maintaining the source code and database of YCweather.
\end{enumerate}

\msusection{Bug Reporting}
YCweather is under development, as such problems and bugs are expected. Issues may be reported using the YCweather \href{https://github.com/aeslaughter/YCweather}{github}\footnote{\href{https://github.com/aeslaughter/YCweather}{\nolinkurl{https://github.com/aeslaughter/YCweather}}} site, please explain the problem in detail. This interface may also be used to make feature suggestions.