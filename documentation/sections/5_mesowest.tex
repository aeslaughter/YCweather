\msusection{MesoWest Data}\label{sec:mesowest}
YCweather is capable of interfacing with weather data archived with MesoWest (\href{http://mesowest.utah.edu/index.html}{\nolinkurl{mesowest.utah.edu}}).  First, a text file names \texttt{mesowest.txt} must be present in the season folder within the YCweather database, see Section \ref{sec:database} for details regarding the folder structure.  This file contains three columns of comma separated data: the first column contains a list of station identifiers as shown on \href{http://mesowest.utah.edu/index.html}{MesoWest}.  For example, YLWM8 is the identifier for the Yellow Mule station in \href{http://mesowest.utah.edu/cgi-bin/droman/mesomap.cgi?state=MT&rawsflag=3}{Southwest Montana}.  The second column contains the desired name of the station (e.g., ``Yellow Mule'' for the above example). The third column contains a corresponding group name associated with the station identifier in the same row.  For example, the Yellow Mule station mentioned is a part of the RAWS network, thus an appropriate name may be ``RAWS Stations'' and perhaps another group would include the National Weather Service stations (e.g., ``NWS Stations'').

The data available for download from MesoWest is somewhat limited, as such the data is not actually downloaded until you select the station in the Program Control window. At this point YCweather will download the data based on the date range specified in the Program Control. Thus, if the data range is altered the station will be need to be reselected if the date range as been altered outside of the range previously selected. Also, the Data List will also need to be recreated. All of the MesoWest data may be updated via the Toolbar button or using the Data menu (see Section \ref{sec:data}).

