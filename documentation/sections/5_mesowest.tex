\msusection{MesoWest Data}\label{sec:mesowest}
YCweather is capable of interfacing with weather data archived with MesoWest (\href{http://mesowest.utah.edu/index.html}{\nolinkurl{mesowest.utah.edu/index.html}}).  First, a text file names \texttt{mesowest.txt} must be present in the season folder within the YCweather database, see Section \ref{sec:database} for details regarding the folder structure.  This file contains two columns of comma separated data: the first column contains a list of station identifiers as shown on \href{http://mesowest.utah.edu/index.html}{MesoWest}.  For example, YLWM8 is the identifier for the Yellow Mule station in \href{http://mesowest.utah.edu/cgi-bin/droman/mesomap.cgi?state=MT&rawsflag=3}{Southwest Montana}.  The second column contains a corresponding group name associated with the station identifier in the same row.  For example, the Yellow Mule station mentioned is a part of the RAWS network, thus an appropriate name may be ``RAWS Stations'' and perhaps another group would include the National Weather Service stations (e.g., ``NWS Stations'').

The data available for download from MesoWest is limited to 30 days, as such if changes to range on the Program Control is altered the MesoWest data may require updating.  When YCweather opens and MesoWest data is desired, it downloads the data in the date range, by default this is the last 48 hours of the data.  The MesoWest data does not update automatically, but can easily be updated via the Toolbar button or using the Data menu (see Section \ref{sec:data}).

